%
% Szakdolgozat minta az Eszterházy Károly Egyetem
% matematika illetve informatika szakos hallgatóinak.
%

\documentclass[
% opciók nélkül: egyoldalas nyomtatás, elektronikus verzió
% twoside, % kétoldalas nyomtatás
% tocnopagenum,% oldalszámozás a tartalomjegyzék után kezdődik
]{thesis-ekf}
\usepackage[T1]{fontenc}
\PassOptionsToPackage{defaults=hu-min}{magyar.ldf}
\usepackage[magyar]{babel}
\usepackage{mathtools,amssymb,amsthm,hyperref,listingsutf8,xcolor,caption,pdfpages}
\usepackage{subcaption}
\lstset{
	inputencoding=utf8,
	language=C,
	basicstyle=\footnotesize,
	breaklines,
	captionpos=b,
	postbreak=\hbox{$\color{red}\hookrightarrow\ $},
	xleftmargin=1cm,
	xrightmargin=1cm,
	backgroundcolor=\color{gray!30},
	frame=tlbr,
	framesep=3pt,
	keywordstyle=\bfseries\color{green!40!black},
	commentstyle=\itshape\color{purple!40!black},
	identifierstyle=\color{blue},
	stringstyle=\color{brown},
	rulecolor=\color{black},
	showstringspaces=false
}

\captionsetup{compatibility=false}
\footnotestyle{rule=fourth}

\newtheorem{tetel}{Tétel}[chapter]
\theoremstyle{definition}
\newtheorem{definicio}[tetel]{Definíció}
\theoremstyle{remark}
\newtheorem{megjegyzes}[tetel]{Megjegyzés}

\begin{document}
\institute{Matematikai és Informatikai Intézet}
\title{Okos otthon hub és irányítóközpont}
\author{Lovász Ákos\\Programtervező informatikus BSc}
\supervisor{Dr. Tajti Tibor\\Egyetemi adjunktus}
\city{Eger}
\date{2021}
\maketitle
\tableofcontents


\chapter*{Bevezetés}
Tanulmányaim folyamán számos technológiával ismerkedtem meg, melyek mindegyike rengeteg
lehetőséget tárt fel előttem, viszont a szakmai gyakorlatom során kiemelkedően megragadta a fantáziámat az Andoid
fejlesztés és a hardverprogramozás összekapcsolása által kialakult rendszerek lehetősége.
\par
Az Android alkalmazások fejlesztése iránt mindig is érdeklődtem, egy-egy kisebb alkalmazást gyakorlásként
már készítettem ezt megelőzően, de komolyabban itt kezdtem vele foglalkozni, megismerkedni a vele járó
sajátosságokkal.
\par
Az ilyen jellegű eszközök kapcsolata és kommunikációja már korai gondolataimban is az okos otthonok felépítésére
emlékeztetett, ezért is gondoltam megfelelő táma választásnak.
\par
A döntést követő kutatás során szembetűnő hátránya volt az okos otthon rendszereknek, hogy a legtöbb ,,márkás''
megoldás elsősorban drága és csak felületes hozzáférést tesznek lehetővé, melyet teljes mértékben a rendszer
gyártója határoz meg. 
\par
Az alternatív, olcsóbb rendszerek bár nyíltabb hozzáállással próbálnak előnyt szerezni,
viszont sokszor erősen a technikai oldalába mélyednek, így egy átlagos felhasználónak bonyolultnak, 
nehezen kezelhetőnek tűnhetnek. Ezen felül gyakran futhatunk olyan problémába, hogy az általunk választott
rendszerben lévő hiányosságokat csak más gyártótól származó eszköz nyújtana megoldást, viszont különböző
gyártók eszközei nagyon ritkán kompatibilisek egymással.
\par
Ezeket az észrevételeket figyelembe véve egyértelműnek tűnt, hogy van lehetőség egy olyan rendszer kivitelezésére,
ami elsősorban olcsóbb, de ugyanakkor nem túlbonyolított, felhasználóbarát marad. Fontos a nyitottság, a bővíthetőség,
és a széleskörű kompatibilitás lehetősége, hogy a felhasználó biztos lehessen abban, hogy a jövőben felmerülő
hiányosságok egyszerűen pótolhatók.

\chapter{A rendszer alapjai}
backend hardver, frontend android, node red, mqtt, bővíthető, könnyen karbantartható stb stb
\section{A kiszolgáló hardver}
OrangePY, azon mosquitto, nodered.
\section{Az Android alkalmazás}
felület, kommunikáció, beállítások
\section{Node-RED}
kommunikáció kezelése, üzenetfeldolgozás, felületes nodered bemutató
\section{MQTT}
hálózati kapcsolat management, felületes mqtt bemutató
\section{Okos eszközök}
kis eszköz amivel lehet okos eszközt szimulálni
(vagy intergrálásával akár készíteni persze)


\chapter{Hardver}
orangepy, bármi android telefon, okos eszközök (esp32)
\section{Orange Pi Zero}
olyan mint a raspberry csak olcsóbb
\section{Okos eszközök}
esp32

\chapter{Szoftver}
nodered, mqtt, android
\section{Node-RED}
kifejtés az aktuális rendszerről
\section{MQTT}
kifejtés az aktuális rendszerről
\section{Android}
kifejtés az aktuális rendszerről
\section{Tesztelés}

\chapter{A renszer működése}
Itt írom le a kész rendszer működését, 
\section{Első indításra felkészítés}
\section{Telefon csatlakoztatása kiszolgálóhoz}
\section{Okos eszközök csatlakoztatása kiszolgálóhoz}
\section{Okos eszközök kezelése az alkalmazásban}

\chapter{Továbbfejlesztési lehetőségek}
user access? felület új eszköztípusok felvételéhez?
cloud service for out of home control

\chapter*{Köszönetnyilvánítás}
\par
Köszönöm a vscode-nak hogy van,
\par
Köszönöm magyarországnak hogy jobban teljesít
\par
Köszönöm a covidnak, hogy átmentem nummatból

\begin{thebibliography}{2}
\end{thebibliography}

\end{document}